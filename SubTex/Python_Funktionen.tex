\section{Python Funktionen}
Dieses Kapitel befasst sich mit Python-Standart-Funktionen.

\subsection{Typen}
Da Python Typen nicht implizit umwandeln kann,müssen sie explizit umgewandelt werden.
\begin{itemize}
\item[list] wandelt iterierbare Datentypen in eine Liste um
\item[tuple] wandelt iterierbare Datentypen in ein Tupel um
\item[string] wandelt Datentypen in Strings um (bei Listen u.ä. werden die
Entsprechenden Klammern und Kommatas mit übernommen)
\item[int] wandelt (Strings und Floats) in Integer um (Strings müssen zahlen mit Basis 10 sein)
\item[float] wandelt (Strings und Integer) in Floats um (Strings müssen Basis 10 sein und wenn mit Nachkommastellen, muss es mit einem Dezimalpunkt getrennt werden
 \item[set] wandelt in sets um
 \item[frozenset] wandelt in frozensets um
 \item[bin] wandelt Integer in ein binären String der Form \kom{0b\dots} um
 \item[oct] wandelt Integer in ein oktalen String der Form \kom{0o\dots} um
 \item[hex] wandelt Integer in ein hexadezimalen String der Form \kom{0x\dots} um
 \item[len] gibt die Anzahl der Elemente von Iterierbaren Datensätzen zurück (\kom{nur die erste Stufe})
\end{itemize}

\subsection{IO}
\begin{itemize}
\item[print] schreibt in die Konsole
\item[input] gibt einen String zurück mit der Konsoleneingabe, der Wert in der Klammer wird zuerst in die Konsole geschrieben
\lstinputlisting[language=Python, firstline=78, lastline=78]{SubTex/Code/DatenTypen.py}
\end{itemize}

\subsection{\kom{range}}
Die range()-Funktion kreirt ein range-Objekt, welches iterierbar bar ist. Mit dieser
Funktion lassen sich sehr gut Listen initialisieren.
\lstinputlisting[language=Python, firstline=83, lastline=85]{SubTex/Code/DatenTypen.py}
Start ist inklusive, stop ist exklusive. Step gibt die Schrittweite an, der Standart ist 1. Kann auch in
negativer richtung (mit einem minus) durchlaufen werden (start und stop müssen angepasst werden).

