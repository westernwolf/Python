\section{Datentypen}
\begin{tabular}{ lll }
	\hline
	\textbf{Datentyp} & \textbf{Beschreibung} & \textbf{False-Wert}\\
	\hline
	NoneType & Indikator für nichts, keinen Wert & None \\
	\hline
	\multicolumn{3}{l}{\textbf{Numerische Datentypen}}\\
	int&Ganze Zahlen & 0\\
	float&Gleitkommazahlen&0.0\\
	bool&Boolesche Werte&False\\
	complex&komplexe Zahlen& 0 + 0j\\
	\hline
	\multicolumn{3}{l}{\textbf{Sequenzielle Datentypen}}\\
	str&Zeichenketten oder Strings&' '\\
	list&Listen (veränderlich)&[]\\
	tuple&Tupel(unveränderlich)&()\\
	bytes&Sequenz von Bytes (unveränderlich)&b' '\\
	bytearray&Sequenz von Bytes (veränderlich)&bytearray(b' ')\\
   \hline
   \multicolumn{3}{l}{\textbf{Mengen}}\\
   set&Menge mit einmalig vorkommenden Objektion&set()\\
   frozenset& Wie set jedoch unveränderlich&frozenset()\\
   \hline
   \multicolumn{3}{l}{\textbf{Assoziative Datentypen}}\\
   dict&Dictionary (Schlüssel-Wert-Paare)&\{\}\\
   \hline
\end{tabular}

Nummerische Datentypen können "nur" Zahlenwerte annehmen.\\
Sequenzielle Datentypen sind vergleichbar mit Arrays. Werte können mehrmals vorkommen
und jeder einzelne besitzt einen spezifischen Index.\\
Mengen sind vergleichbar mit mathematischen Mengen, jedes Element ist einzigartig
und die Reihenfolge ist willkürlich.\\
Assoziative Datentypen sind ähnlich wie die Sequenziellen Datentypen, jedoch besitzen sie 
keinen nummerischen Index, aber Keys aus (unveränderlichen) Datentypen.